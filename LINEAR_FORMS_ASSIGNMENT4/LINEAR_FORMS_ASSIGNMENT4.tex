\documentclass[journal,12pt,twocolumn]{IEEEtran}

\usepackage{graphicx}
\usepackage{setspace}
\usepackage{gensymb}
\singlespacing
\usepackage[cmex10]{amsmath}
\usepackage{amssymb}
\usepackage{xurl}
\usepackage{tabularx}
\usepackage{amsthm}
\usepackage{comment}
\usepackage{mathrsfs}
\usepackage{txfonts}
\usepackage{stfloats}
\usepackage{bm}
\usepackage{cite}
\usepackage{cases}
\usepackage{subfig}
\usepackage{arydshln}
\usepackage{longtable}
\usepackage{multirow}

\usepackage{enumitem}
\usepackage{mathtools}
\usepackage{steinmetz}
\usepackage{tikz}
\usepackage{circuitikz}
\usepackage{verbatim}
\usepackage{tfrupee}
\usepackage[breaklinks=true]{hyperref}
\usepackage{graphicx}
\usepackage{tkz-euclide}
\usetikzlibrary{automata, positioning}
\usetikzlibrary{calc,math}
\usepackage{listings}
    \usepackage{color}                                            %%
    \usepackage{array}                                            %%
    \usepackage{longtable}                                        %%
    \usepackage{calc}                                             %%
    \usepackage{multirow}                                         %%
    \usepackage{hhline}                                           %%
    \usepackage{ifthen}                                           %%
    \usepackage{lscape}     
\usepackage{multicol}
\usepackage{chngcntr}
\usepackage{blkarray}

\DeclareMathOperator*{\Res}{Res}

\renewcommand\thesection{\arabic{section}}
\renewcommand\thesubsection{\thesection.\arabic{subsection}}
\renewcommand\thesubsubsection{\thesubsection.\arabic{subsubsection}}

\renewcommand\thesectiondis{\arabic{section}}
\renewcommand\thesubsectiondis{\thesectiondis.\arabic{subsection}}
\renewcommand\thesubsubsectiondis{\thesubsectiondis.\arabic{subsubsection}}


\hyphenation{op-tical net-works semi-conduc-tor}
\def\inputGnumericTable{}                                 %%

\lstset{
%language=C,
frame=single, 
breaklines=true,
columns=fullflexible
}
\begin{document}


\newtheorem{theorem}{Theorem}[section]
\newtheorem{problem}{Problem}
\newtheorem{proposition}{Proposition}[section]
\newtheorem{lemma}{Lemma}[section]
\newtheorem{corollary}[theorem]{Corollary}
\newtheorem{example}{Example}[section]
\newtheorem{definition}[problem]{Definition}

\newcommand{\BEQA}{\begin{eqnarray}}
\newcommand{\EEQA}{\end{eqnarray}}
\newcommand{\define}{\stackrel{\triangle}{=}}
\bibliographystyle{IEEEtran}
\raggedbottom
\setlength{\parindent}{0pt}
\providecommand{\mbf}{\mathbf}
\providecommand{\pr}[1]{\ensuremath{\Pr\left(#1\right)}}
\providecommand{\qfunc}[1]{\ensuremath{Q\left(#1\right)}}
\providecommand{\sbrak}[1]{\ensuremath{{}\left[#1\right]}}
\providecommand{\lsbrak}[1]{\ensuremath{{}\left[#1\right.}}
\providecommand{\rsbrak}[1]{\ensuremath{{}\left.#1\right]}}
\providecommand{\brak}[1]{\ensuremath{\left(#1\right)}}
\providecommand{\lbrak}[1]{\ensuremath{\left(#1\right.}}
\providecommand{\rbrak}[1]{\ensuremath{\left.#1\right)}}
\providecommand{\cbrak}[1]{\ensuremath{\left\{#1\right\}}}
\providecommand{\lcbrak}[1]{\ensuremath{\left\{#1\right.}}
\providecommand{\rcbrak}[1]{\ensuremath{\left.#1\right\}}}
\theoremstyle{remark}
\newtheorem{rem}{Remark}
\newcommand{\sgn}{\mathop{\mathrm{sgn}}}
\providecommand{\abs}[1]{\vert#1\vert}
\providecommand{\res}[1]{\Res\displaylimits_{#1}} 
\providecommand{\norm}[1]{\lVert#1\rVert}
%\providecommand{\norm}[1]{\lVert#1\rVert}
\providecommand{\mtx}[1]{\mathbf{#1}}
\providecommand{\mean}[1]{E[ #1 ]}
\providecommand{\fourier}{\overset{\mathcal{F}}{ \rightleftharpoons}}
%\providecommand{\hilbert}{\overset{\mathcal{H}}{ \rightleftharpoons}}
\providecommand{\system}{\overset{\mathcal{H}}{ \longleftrightarrow}}
	%\newcommand{\solution}[2]{\textbf{Solution:}{#1}}
\newcommand{\solution}{\noindent \textbf{Solution: }}
\newcommand{\cosec}{\,\text{cosec}\,}
\providecommand{\dec}[2]{\ensuremath{\overset{#1}{\underset{#2}{\gtrless}}}}
\newcommand{\myvec}[1]{\ensuremath{\begin{pmatrix}#1\end{pmatrix}}}
\newcommand{\mydet}[1]{\ensuremath{\begin{vmatrix}#1\end{vmatrix}}}
\newcommand*{\permcomb}[4][0mu]{{{}^{#3}\mkern#1#2_{#4}}}
\newcommand*{\perm}[1][-3mu]{\permcomb[#1]{P}}
\newcommand*{\comb}[1][-1mu]{\permcomb[#1]{C}}
\numberwithin{equation}{subsection}
\makeatletter
\@addtoreset{figure}{problem}
\makeatother
\let\StandardTheFigure\thefigure
\let\vec\mathbf
\renewcommand{\thefigure}{\theproblem}
\def\putbox#1#2#3{\makebox[0in][l]{\makebox[#1][l]{}\raisebox{\baselineskip}[0in][0in]{\raisebox{#2}[0in][0in]{#3}}}}
     \def\rightbox#1{\makebox[0in][r]{#1}}
     \def\centbox#1{\makebox[0in]{#1}}
     \def\topbox#1{\raisebox{-\baselineskip}[0in][0in]{#1}}
     \def\midbox#1{\raisebox{-0.5\baselineskip}[0in][0in]{#1}}
\vspace{3cm}
\title{\textbf{LINEAR SYSTEMS AND SIGNAL PROCESSING \\ ASSIGNMENT 4}}
\author{GANJI VARSHITHA - AI20BTECH11009}
\maketitle
\newpage
\bigskip
\renewcommand{\thefigure}{\arabic{figure}}
\renewcommand{\thetable}{\arabic{table}}
Download latex codes from 
%
\begin{lstlisting}
https://github.com/VARSHITHAGANJI/EE3900_VECTORS_ASSIGNMENTS/blob/main/LINEAR_FORMS_ASSIGNMENT4/LINEAR_FORMS_ASSIGNMENT4.tex
\end{lstlisting}
Download all python codes from
\begin{lstlisting}
https://github.com/VARSHITHAGANJI/EE3900_VECTORS_ASSIGNMENTS/blob/main/LINEAR_FORMS_ASSIGNMENT4/skew_lines_code.py
\end{lstlisting}
\section*{QUESTION}
\textbf{Linearforms 2.23}
 Find the shortest distance between the lines
\begin{align}
\label{eq:1}
 L_{1}: \; \Vec{x} ={}& \myvec{1 \\ 2\\ 3 } +\lambda_{1}\myvec{1 \\ -3 \\ 2}\\
 \label{eq:2}
 L_{2}: \; \Vec{x} ={}& \myvec{4 \\ 5\\ 6} + \lambda_{2}\myvec{2 \\ 3 \\ 1} 
\end{align}
\section*{SOLUTION}
We have,
\begin{align}
    \label{eq:3}                       
    L_{1}: \: \Vec{x}={}&\Vec{a_{1}}+\lambda_{1}\Vec{b_{1}}\\
    \label{eq:4}
    L_{2}: \: \Vec{x}={}&\Vec{a_{2}}+\lambda_{2}\Vec{b_{2}}
\end{align}
where $\Vec{a_{i}},\Vec{b_{i}}$ are positional vector, slope vector of line $L_{i}$ respectively.\\
As $\Vec{b_{1}}\neq k \Vec{b_{2}}$, the lines are not parallel to each other.
\\ Let us assume that $L_{1}$ and $L_{2}$ intersect at a point. Therefore,
\begin{align}
    \label{eq:5}
    \myvec{1 \\ 2\\ 3 } +\lambda_{1}\myvec{1 \\ -3 \\ 2} = \myvec{4 \\ 5\\ 6} + \lambda_{2}\myvec{2 \\ 3 \\ 1} \\
    \label{eq:6}
    \lambda_{1}\myvec{1 \\ -3 \\ 2} + \lambda_{2}\myvec{-2 \\ -3 \\ -1}= \myvec{3 \\ 3\\ 3 } \\
    \label{eq:7}
    \myvec{1 & -2\\ -3 &-3 \\ 2 & -1}\myvec{\lambda_{1}\\ \lambda_{2}}=
    \myvec{3 \\ 3\\ 3 }
\end{align}
The augmented matrix for the above equation in row reduced form
\begin{multline}
    \myvec{1 & -2 & 3\\ -3 &-3 & 3\\ 2 & -1 & 3} \xleftrightarrow{R_{2}\leftarrow R_{2}+3R_{1}}
     \myvec{1 & -2 & 3\\ 0 &3 & 9\\ 2 & -1 & 3}\\
     \xleftrightarrow{R_{3}\leftarrow R_{3}-2R_{1}}
     \myvec{1 & -2 & 3\\ 0 &3 & 9\\ 0 & 3 & -3}
   \xleftrightarrow[R_{3}\leftarrow R_{3}-3R_{2}]{R_{2}\leftarrow\frac{R_{2}}{3}}
    \myvec{1 & -2 & 3\\ 0 & 1 & 9\\ 0 & 0 & -3}
\end{multline}
$\therefore$ The rank of the matrix = 3. Hence the lines do not intersect. \\
$L_{1}$ and $L_{2}$ are skew lines. \\
Let d be the shortest distance and $\Vec{p_{1}}, \Vec{p_{2}}$ be positional vectors of its end points.
For d to be shortest, we know that,
\begin{align}
    \label{eq:9}
    \Vec{b_{1}}^\top\brak{\Vec{p_{2}}-\Vec{p_{1}}}=0\\
    \label{eq:10}
     \Vec{b_{2}}^\top\brak{\Vec{p_{2}}-\Vec{p_{1}}}=0\\
     \label{eq:11}
     \Vec{b_{1}}^\top\brak{\brak{\Vec{a}_{2} - \Vec{a}_{1}}}+\myvec{\Vec{b_{2}} & \Vec{b}_{1}}\myvec{\lambda_{1} \\ \lambda_{2}}\\
     \label{eq:12}
     \Vec{b_{2}}^\top\brak{\brak{\Vec{a}_{2} - \Vec{a}_{1}}}+\myvec{\Vec{b_{2}} & \Vec{b_{1}}}\myvec{\lambda_{1} \\ \lambda_{2}}
\end{align}
Let 
\begin{align}
\label{eq:13}
    \Vec{B}&=\myvec{\Vec{b_{2}} & \Vec{b}_{1}} & \Vec{B}^\top&=\myvec{\Vec{b_{2}}^\top \\ \Vec{b_{1}}^\top}
\end{align}
By combining equations \eqref{eq:11} and \eqref{eq:12} and writing in terms of $\Vec{B}$ and $\Vec{B}^\top$ using \eqref{eq:13}, we get
\begin{align}
    \label{eq:14}
    \Vec{B}^\top\Vec{B}\myvec{\lambda_{1} \\ -\lambda_{2}}= \Vec{B}^\top\brak{\Vec{a}_{1} - \Vec{a}_{2}}
\end{align}
By putting the values of $a_{1},a_{2},b_{1},b_{2}$ in \eqref{eq:14}, we get
\begin{align}
    \label{eq:15}
    \myvec{14 & -5 \\ -5 & 14}\myvec{\lambda_{1} \\ -\lambda_{2}}=\myvec{-18 \\ 0}
\end{align}
Solving \eqref{eq:15}, we get
\begin{align}
    \label{eq:16}
    \myvec{\lambda_{1} \\ -\lambda_{2}}=\myvec{-1.4736 \\ -0.5263}
\end{align}
Substituting the value of $\lambda_{1}$ and $\lambda_{2}$ in \eqref{eq:3} and \eqref{eq:4}, we get
\begin{align}
    \label{eq:17}
    \Vec{p_{1}} &= \myvec{-0.4736 \\ 6.4210 \\ 0.0526 }   &    \Vec{p_{2}}&=\myvec{ 5.05263\\ 6.57894\\ 6.5263}
\end{align}
Hence, the shortest distance between these two skew lines is
\begin{align}
    \label{eq:18}
    d = \norm{\Vec{p_{2}}-\Vec{p_{1}}} = 8.5131
\end{align}
\begin{figure}[h]
\centering
\includegraphics[width=\columnwidth]{q4}
\caption{Plot of skew lines}
\end{figure}
\end{document}
