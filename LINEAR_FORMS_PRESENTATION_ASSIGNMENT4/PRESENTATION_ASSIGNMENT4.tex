\documentclass{beamer}
\usepackage{listings}
\lstset{
%language=C,
frame=single, 
breaklines=true,
columns=fullflexible
}
\usepackage{blkarray}
\usepackage{subcaption}
\usepackage{url}
\usepackage{tikz}
\usepackage{tkz-euclide} % loads  TikZ and tkz-base
%\usetkzobj{all}
\usetikzlibrary{calc,math}
\usepackage{float}
\providecommand{\brak}[1]{\ensuremath{\left(#1\right)}}
\providecommand{\pr}[1]{\ensuremath{\Pr\left(#1\right)}}
\newcommand{\myvec}[1]{\ensuremath{\begin{pmatrix}#1\end{pmatrix}}}
\newcommand\norm[1]{\left\lVert#1\right\rVert}
\renewcommand{\vec}[1]{\mathbf{#1}}
\usepackage[export]{adjustbox}
\usepackage[utf8]{inputenc}
\usepackage{amsmath}
\usepackage{tikz}
\usetikzlibrary{automata, positioning}
\usetheme{Boadilla}




\title{Assignment 4 - Linear Forms Q2.23}
\author{Ganji Varshitha - AI20BTECH11009}
\date{\today }
\begin{document}

\begin{frame}
\titlepage
\end{frame}

\begin{frame}
\frametitle{Question}
\begin{block}{Linear Forms Q2.23}
Find the shortest distance between the lines
\begin{align}
\label{eq:1}
 L_{1}: \; \Vec{x} ={}& \myvec{1 \\ 2\\ 3 } +\lambda_{1}\myvec{1 \\ -3 \\ 2}\\
 \label{eq:2}
 L_{2}: \; \Vec{x} ={}& \myvec{4 \\ 5\\ 6} + \lambda_{2}\myvec{2 \\ 3 \\ 1} 
\end{align}
\end{block}
\end{frame}

\begin{frame}
\frametitle{Solution}
\begin{block}{Prerequisites}
The general equation of a line in 3D plane can be written as :
\begin{align}
\Vec{x}= \Vec{a}+\lambda\Vec{b} \label{eq:3}  
\end{align}
where $\Vec{a}$ and $\Vec{b}$ are positional vector and slope vector of the line respectively.\\
Lines can be intersecting or parallel if they are coplanar and non intersecting and non parallel if they are not coplanar 
\end{block}
\begin{block}{Skew Lines}
Skew lines are two lines that do not intersect and are not parallel.\\
Example: Pair of lines through opposite edges of a regular tetrahedron.
\end{block}

\end{frame}

\begin{frame}
\frametitle{Solution Contd.}
The lines $L_{1}$ and $L_{2}$ are not parallel as $\Vec{b_{1}}\neq k\Vec{b_{2}}$.\\
Let the given lines $L_{1}$ and $L_{2}$ in the form of $\Vec{a_{i}}+\lambda_{i}\Vec{b_{i}}$ be intersecting, then
\begin{align}
    \label{eq:4}
    \myvec{1 \\ 2\\ 3 } +\lambda_{1}\myvec{1 \\ -3 \\ 2} ={}& \myvec{4 \\ 5\\ 6} + \lambda_{2}\myvec{2 \\ 3 \\ 1} \\
    \label{eq:5}
    \myvec{1 & -2\\ -3 &-3 \\ 2 & -1}\myvec{\lambda_{1}\\ \lambda_{2}}={}&\myvec{3 \\ 3\\ 3 }
\end{align}
The augmented matrix for \eqref{eq:5} in row reduced form becomes
\begin{align}
\label{eq:6}
\myvec{1 & -2 & 3\\ -3 &-3 & 3\\ 2 & -1 & 3} \longleftrightarrow \myvec{1 & -2 & 3\\ 0 & 1 & 9\\ 0 & 0 & -3}
\end{align}
Since the rank of the augmented matrix is 3, the system of equations is inconsistent.\\
Hence, the lines are not intersecting.
\end{frame}
\begin{frame}
Since the lines are neither parallel nor intersecting, the lines are said to be skew lines.
\begin{block}{Finding shortest distance between two skew lines}
Let $\Vec{p_{1}}$, $\Vec{p_{2}}$ be the closest points on lines $L_{1}$ and $L_{2}$ respectively.\\
Then the shortest distance between two skew lines will be the length of line perpendicular to both the lines $L_{1}$, $L_{2}$ and passing through $\Vec{p_{1}}$ and $\Vec{p_{2}}$.\\
The slope of line passing through $\Vec{p_{1}}$ and $\Vec{p_{2}}$ is along $\Vec{p_{2}}-\Vec{p_{1}}$, which is perpendicular to both $L_{1}$ and $L_{2}$. Thus,
\begin{align}
    \label{eq:7}
    \Vec{b_{1}}^\top\brak{\Vec{p_{2}}-\Vec{p_{1}}}={}&0\\
    \label{eq:8}
     \Vec{b_{2}}^\top\brak{\Vec{p_{2}}-\Vec{p_{1}}}={}&0
\end{align}
Let $\Vec{B}=\myvec{\Vec{b_{2}} & \Vec{b}_{1}}$, combining \eqref{eq:7} and \eqref{eq:8} in terms of $\Vec{B}$ and $\Vec{B}^\top$, we have
\begin{align}
    \label{eq:9}
    \Vec{B}^\top\Vec{B}\myvec{\lambda_{2} \\ -\lambda_{1}}= \Vec{B}^\top\brak{\Vec{a}_{1} - \Vec{a}_{2}}
\end{align} 
\end{block}
\end{frame}
\begin{frame}
Substituting values of $a_{1}$, $a_{2}$, $b_{1}$, $b_{2}$, in \eqref{eq:9}
\begin{align}
    \label{eq:10}
    \myvec{14 & -5 \\ -5 & 14}\myvec{\lambda_{2} \\ -\lambda_{1}}=\myvec{-18 \\ 0}
\end{align}
Solving for $\lambda_{1}$ and $\lambda_{2}$, 
\begin{align}
    \label{eq:11}
    \myvec{\lambda_{2} \\ -\lambda_{1}}=\myvec{-1.4736 \\ -0.5263}
\end{align}
The closest points are
\begin{align}
    \label{eq:12}
    \Vec{p_{1}} &= \myvec{1.5263\\ 0.4210 \\ 4.0526 }   &    \Vec{p_{2}}&=\myvec{ 1.0526\\0.5789\\4.5263}
\end{align}
Therefore, the shortest distance between these two skew lines is
\begin{align}
    \label{eq:18}
    d = \norm{\Vec{p_{2}}-\Vec{p_{1}}} = 0.6882
\end{align}
\end{frame}

\end{document}